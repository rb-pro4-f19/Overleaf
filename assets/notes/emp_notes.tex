\section{Embedded Software Design}
1. Processes vs Threads
- Limited Memory vs Synkronisering Extras
    - Limiited memory therfor Threads are chosen. -- ...
    - No nested processes (no threads in processes)
- Independent Units vs Event Driven Programs
    - Indepedent units, because a event driven programs are hard to realise as RTOS.
    - Small project
    - Low requirement specifications

2. Thread usages
- Multi-threading and Parallel programming
    - Our CPU supports only 1 "real" thread at a time.
    - We do need / chosen several Threads in this project to realise it.
- Hand-in-hand blocking \& execution
    - We need systemblocking as it is better than to race.
- Code \& Libraries to be re-entrant \& thread-safe
    - It is required thread-safe.
- PThread Library
  - Model cases
    - Producer - Consumer --
    - Peer ( multiple I/O processing) -- SPI take care
    - Need to choose a Library -
    - FREERTOS.org - https://www.freertos.org/RTOS.html

3. Scheduler \& Prio
- Soft Real Time vs Hard Real time
    - Choose which events or processes or tasks are which
    - Nice vs Prio
- Time slicing vs Starvation
    - RTOS tasks are to be run.. but soft real time might consider time slicing due to starvation.
- Sync in Priobased Scheduling
    - ....
- Deadlock Avoidance (bankers)
    - Code must be tested for deadlocks
- Priority Inversion - prio inheritance, prio protect
    - Might not happend, we will design so this will not occur otherwise library for this..

5. Signals vs polling
    - Small systems no need for signals - and indepedent units - therefor polling
- Async vs Sync
    - Async UART, SPI,
- Racing vs Blocking - A race condition is a flaw that occurs when the timing or ordering of events affects
    - Blocking
- Timers vs loops
    - Timers if nes otherwise loops.

6. Interrupts vs Signals
    - No interrupts in generel bcs no reason to use it!

7. Sockets
- Interface to multiple Devices
    - 1 UART to PC
    - 3 SPI per purpose FPGA.
    - Joystick
        - But no common bus

8. Fejl Tolerancer
- Default ErrorHandler / ExceptionHandler
    - blinking LED + UART code
- Avoid Race Conditions w/ proper sync
- Avoid Memory Leaks w/ Memory usage balancing - Verify memory profilers
- Handle error cases, Extreme conditions
    - Defaults for a switch, else, if
- Avoid infinite  loops - use timeouts.
- Regorganize code to avoid / prevent deadlocks

9. Design for Crash Recovery
- Restart crusial flows - Main task f.eks.
- Design for Hang Recovery - with Watchdogs
- Design for power fialtures - brown out detections / power-down saves

Design for Hardware
- Speed Mismatch (faster/slower)
- Time-sensitive Hardware interaction spi, i2s - precise busy wait delays
Non-responsive hardware
- Timeout rather than uncondictional accesses
 Hardware with irregular/incorrect responses
- Do prober handling for the extreme / abnormal / invalid cases
Hardware delays (Debounce, Rise time)
- Appropiate timers / delays to offset

Design for Performance
- IPC vs Sync
    - Multiprocesses (Context Switches) vs Multihreaded ( same context)
    - Communication vs Sync overheads
- Schedulers and Priority
    - Normal vs Real time
    - Nice vs Priority
- Yielding vs Busy wait
    - LAtency vs Precision
- Interrupt / signal usage
    - High Prio and Async attention
    - Missing = invalidate assumptions
- Design for Real time Performance
- Have very few real time processes
- Keep the real time short \& effiecient
    - INGEN MEMORY ALLOCATION
    - if essential move to init sections
    