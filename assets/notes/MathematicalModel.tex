% Language
\usepackage[utf8]{inputenc}
\usepackage[english]{babel}
\babelhyphenation[english]{feed-back-pro-ces-ser}

% Page Formatting
\usepackage[a4paper, width=150mm, top=25mm, bottom=25mm, includefoot]{geometry}

% Line Height
\usepackage{setspace}
\linespread{1.3}

% Paragraph Indentation
\setlength{\parindent}{0pt}

% Captions
\usepackage{caption}
\usepackage{subcaption}

% Header
\usepackage{fancyhdr}
\pagestyle{fancy}
\lhead{}
\chead{}
\makeatletter
\rhead{\slshape{\nouppercase{\rightmark}}}
\makeatother

% Footnote
\usepackage[bottom]{footmisc}
\emergencystretch=1em
\usepackage{scrextend}
\addtokomafont{labelinglabel}{\textnormal}
\usepackage{pdfpages}
\usepackage{needspace}

% Math & Symbols
\usepackage{float}
\usepackage{amsmath}
\usepackage{siunitx}
\usepackage{esvect}
\usepackage{verbatim}

% Equation Formatting
\setlength{\abovedisplayskip}{1em}
\setlength{\belowdisplayskip}{1em}

% Enumeration & Tables
\usepackage{enumerate}
\usepackage{enumitem}
\usepackage{multirow} 
\usepackage{multicol}
\usepackage{ltablex}
\usepackage{spreadtab}
\usepackage{booktabs}
\usepackage{tabto}

% Enumeration Formatting
\newcommand{\tabitem}{~~\llap{\textbullet}~~}
\newcommand{\sqrbulletsml}{\textcolor{black}{\raisebox{.45ex}{\rule{.6ex}{.6ex}}}}
\newcommand{\sqrbulletmed}{\textcolor{black}{\raisebox{.40ex}{\rule{.7ex}{.7ex}}}}

\renewcommand{\labelitemi}{\sqrbulletmed}
\renewcommand{\labelitemii}{\sqrbulletsml}
\renewcommand{\labelitemiii}{\sqrbulletsml}
\renewcommand{\labelitemiv}{\sqrbulletsml}

% Colors
\usepackage{xcolor}

% Color Definitions (latexcolor.com)
\definecolor{cerulean}{rgb}{0.0, 0.48, 0.65}
\definecolor{earthyellow}{rgb}{0.88, 0.66, 0.37}
\definecolor{darkmagenta}{rgb}{0.55, 0.0, 0.55}
\definecolor{darkolivegreen}{rgb}{0.33, 0.42, 0.18}
\definecolor{codegray}{gray}{0.9}
%\newcommand{\code}[1]{\colorbox{codegray}{\texttt{#1}}}

% Code
\usepackage{algorithm}
\usepackage[noend]{algpseudocode}
\usepackage{listings}
\renewcommand{\arraystretch}{1.2}
\renewcommand{\tabcolsep}{0.2cm}

% Code Snippet Formatting
\lstset{
    backgroundcolor=\color{black!5},        % set backgroundcolor
    basicstyle=\footnotesize\ttfamily,      % basic font setting
    frame=single,                           % draw a frame at the top and bottom of the code block
    framesep=5pt,                           % frame margin
    xleftmargin=5pt,                        % frame margin
    xrightmargin=5pt,                       % frame margin
    tabsize=4,                              % tab space width
    showstringspaces=false,                 % don't mark spaces in strings
    breaklines=true,                        % wrap lines
    commentstyle=\color{darkolivegreen},    % comment color
    keywordstyle=\color{darkmagenta},       % keyword color
    stringstyle=\color{earthyellow},        % string color
    identifierstyle=\color{cerulean}
}


% Bibliography & Citation
\usepackage{csquotes}
\usepackage{hyperref}
\usepackage[backend=biber, sorting=none, block=ragged, citestyle=authoryear]{biblatex}
\DeclareNameAlias{sortname}{last-first}
\DeclareNameAlias{default}{last-first}
\addbibresource{resources.bib}

% Continuous Number (1, 2, 3 ..)
\usepackage{chngcntr}
\counterwithout{figure}{chapter}
\counterwithout{table}{chapter}
\counterwithout{equation}{chapter}

% Graphics
\usepackage{graphicx}
\usepackage{pgfplots}  %Enabels draw diagrams
\usepackage{tikz}
\usetikzlibrary{shapes,arrows}

\usetikzlibrary{shapes,arrows}
\tikzstyle{decision} = [diamond, draw, fill=blue!20, 
    text width=4.5em, text badly centered, node distance=3cm, inner sep=0pt]
\tikzstyle{block} = [rectangle, draw, fill=blue!20, 
    text width=5em, text centered, rounded corners, minimum height=4em]
\tikzstyle{line} = [draw, -latex']
\tikzstyle{cloud} = [draw, ellipse,fill=red!20, node distance=3cm,
    minimum height=2em, text badly centered, text width=4.5em]

\makeatother
For regulation of the pan/tilt assembly, a mathematical model needs to be formulated. The model has to describe the movement (speed and acceleration) of the system, based on motor voltage and the resulting torque produced by the motor.



\begin{figure}[H]
    \centering
      \begin{circuitikz} \draw
        (0,0) to [open,v=$V_m$] (0,3)
        to[L = $L$, v<= $L\cdot\frac{dI}{dt}$] (3,3) to[R = $R_1$ ,v<=$R_1 \cdot I$] (6,3) to node[elmech]{M}(6,0)
        -- (0,0)
        ;  
        \draw (6.3,2.5) to [open,v^<=$k_{e}\dot{\theta}_m$] (6.3,0.5);
     \end{circuitikz}
    \caption{Circuit of the DC motor}
    \label{fig:my_label}
\end{figure}

Given the voltages of the circuit elements it is possible to derive a function for $V_m$. Where the expression $k_{e}\dot{\theta}_m$ describes the counter electromotive force of the DC motor.

\begin{equation}
    V_m = L\,\frac{dI}{dt} + R_{1}I(t)+k_{e}\dot{\theta}_m(t)
\end{equation}

From Newton's 2nd Law of rotation the net torque of a rotational system is given by the equation $\tau _{net} = J\cdot \ddot{\theta}$. The forces acting on the motor consists of the torque produced by the DC motor and the force of friction.
The torque produced is given by the equation $\tau = I\cdot k_{e}$, where $k$ is the  electromotive force constant specific to the motor. The following equation describes the net torque of the DC motor.


\begin{equation}
    I(t)k_{m2} = \mu_m\dot{\theta}_m(t) + J_m\ddot{\theta}_m(t)
\end{equation}

In order to derive a mathematical model for the entire assembly it is necessary to incorporate the torque contribution of the remaining pan/tilt system. Two mathematical expressions can be derived; one describing the torque of the pan-system and one describing the tilt-system.
\newline

The force from the motor i transferred to the rotational axes of the pant/tilt-mechanisms with a gear-ratio $G$, this results in a reduced angular velocity, compared to the motor.

The net torque of the tilt-mechanism can be described by the torque of the rotation and a force of friction in addition to the torque contributed by the DC-motor. Hence the following equation for the tilt-mechanism.

\begin{equation}
    I(t)k_{m2} = \mu_m\dot{\theta}_m(t) + J_m\ddot{\theta}_m(t) + \frac{J_{tilt}\ddot{\theta}_m + \mu_{tilt}\dot{\theta}_m}{G}
\end{equation}

The moment of inertia of the pan-mechanism is dependant on the angle of the tilt-mechanisms \ref{}. Therefore the net torque of the pan-mechanism is as follows.

\begin{equation}
    I(t)k_{m2} = \mu_m\dot{\theta}_m(t) + J_m\ddot{\theta}_m(t) + \frac{J_{pan}(\theta_{tilt})\ddot{\theta}_m + \mu_{pan}\dot{\theta}_m}{G}
\end{equation}

