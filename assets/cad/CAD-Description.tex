\documentclass{article}
% Language
\usepackage[utf8]{inputenc}
\usepackage[english]{babel}
\babelhyphenation[english]{feed-back-pro-ces-ser}

% Page Formatting
\usepackage[a4paper, width=150mm, top=25mm, bottom=25mm, includefoot]{geometry}

% Line Height
\usepackage{setspace}
\linespread{1.3}

% Paragraph Indentation
\setlength{\parindent}{0pt}

% Captions
\usepackage{caption}
\usepackage{subcaption}

% Header
\usepackage{fancyhdr}
\pagestyle{fancy}
\lhead{}
\chead{}
\makeatletter
\rhead{\slshape{\nouppercase{\rightmark}}}
\makeatother

% Footnote
\usepackage[bottom]{footmisc}
\emergencystretch=1em
\usepackage{scrextend}
\addtokomafont{labelinglabel}{\textnormal}
\usepackage{pdfpages}
\usepackage{needspace}

% Math & Symbols
\usepackage{float}
\usepackage{amsmath}
\usepackage{siunitx}
\usepackage{esvect}
\usepackage{verbatim}

% Equation Formatting
\setlength{\abovedisplayskip}{1em}
\setlength{\belowdisplayskip}{1em}

% Enumeration & Tables
\usepackage{enumerate}
\usepackage{enumitem}
\usepackage{multirow} 
\usepackage{multicol}
\usepackage{ltablex}
\usepackage{spreadtab}
\usepackage{booktabs}
\usepackage{tabto}

% Enumeration Formatting
\newcommand{\tabitem}{~~\llap{\textbullet}~~}
\newcommand{\sqrbulletsml}{\textcolor{black}{\raisebox{.45ex}{\rule{.6ex}{.6ex}}}}
\newcommand{\sqrbulletmed}{\textcolor{black}{\raisebox{.40ex}{\rule{.7ex}{.7ex}}}}

\renewcommand{\labelitemi}{\sqrbulletmed}
\renewcommand{\labelitemii}{\sqrbulletsml}
\renewcommand{\labelitemiii}{\sqrbulletsml}
\renewcommand{\labelitemiv}{\sqrbulletsml}

% Colors
\usepackage{xcolor}

% Color Definitions (latexcolor.com)
\definecolor{cerulean}{rgb}{0.0, 0.48, 0.65}
\definecolor{earthyellow}{rgb}{0.88, 0.66, 0.37}
\definecolor{darkmagenta}{rgb}{0.55, 0.0, 0.55}
\definecolor{darkolivegreen}{rgb}{0.33, 0.42, 0.18}
\definecolor{codegray}{gray}{0.9}
%\newcommand{\code}[1]{\colorbox{codegray}{\texttt{#1}}}

% Code
\usepackage{algorithm}
\usepackage[noend]{algpseudocode}
\usepackage{listings}
\renewcommand{\arraystretch}{1.2}
\renewcommand{\tabcolsep}{0.2cm}

% Code Snippet Formatting
\lstset{
    backgroundcolor=\color{black!5},        % set backgroundcolor
    basicstyle=\footnotesize\ttfamily,      % basic font setting
    frame=single,                           % draw a frame at the top and bottom of the code block
    framesep=5pt,                           % frame margin
    xleftmargin=5pt,                        % frame margin
    xrightmargin=5pt,                       % frame margin
    tabsize=4,                              % tab space width
    showstringspaces=false,                 % don't mark spaces in strings
    breaklines=true,                        % wrap lines
    commentstyle=\color{darkolivegreen},    % comment color
    keywordstyle=\color{darkmagenta},       % keyword color
    stringstyle=\color{earthyellow},        % string color
    identifierstyle=\color{cerulean}
}


% Bibliography & Citation
\usepackage{csquotes}
\usepackage{hyperref}
\usepackage[backend=biber, sorting=none, block=ragged, citestyle=authoryear]{biblatex}
\DeclareNameAlias{sortname}{last-first}
\DeclareNameAlias{default}{last-first}
\addbibresource{resources.bib}

% Continuous Number (1, 2, 3 ..)
\usepackage{chngcntr}
\counterwithout{figure}{chapter}
\counterwithout{table}{chapter}
\counterwithout{equation}{chapter}

% Graphics
\usepackage{graphicx}
\usepackage{pgfplots}  %Enabels draw diagrams
\usepackage{tikz}
\usetikzlibrary{shapes,arrows}

\usetikzlibrary{shapes,arrows}
\tikzstyle{decision} = [diamond, draw, fill=blue!20, 
    text width=4.5em, text badly centered, node distance=3cm, inner sep=0pt]
\tikzstyle{block} = [rectangle, draw, fill=blue!20, 
    text width=5em, text centered, rounded corners, minimum height=4em]
\tikzstyle{line} = [draw, -latex']
\tikzstyle{cloud} = [draw, ellipse,fill=red!20, node distance=3cm,
    minimum height=2em, text badly centered, text width=4.5em]

\makeatother

\title{CAD-Description}
\author{Christian Skov Esbensen \& Markus Dahl Lauritsen}
\date{February 2019}

\begin{document}
\section{Introduction}
At the start of the project a pan-tilt system was handed out. The system consists of three different aluminium profile assemblies; a stand, an outer pan, and inner tilt assembly, thus giving the system two degrees of freedom. 
A CAD drawing of the system was produced in the software Fusion 360, from measurements taken with a vernier caliper with an accuracy of $ 0.05 \,\mathrm{mm}$.
The drawn parts of the was given the properties of there respective materials, so that a approximation of the weight of each assembly could be made. On the basis of these approximations the software Fusion 360 was able to calculate the moment of inertia for the pan and tilt assembly.

\section{The moment of inertia}
The moment of inertia of the tilt assembly can be calculated easily, because it is not dependent on any variables. The same can not be said about the pan assembly since the inertia of the tilt assembly is included, and dependent on the angle of the tilt system.
\begin{equation}\label{eq:I_tilt}
    I_{Tilt} = 0.01682\, \frac{\mathrm{Kg}}{m^2}
\end{equation}
Fusion 360 were able to generate the different moments of inertia by manipulating the angle of the tilt assembly. These results are shown in table \ref{ta:Inertia_PAN}. Equation \eqref{eq:I_pan} shows a sinusoidal approximation of the pan assembly's moment of inertia, based on the values in table \ref{ta:Inertia_PAN}.

\begin{table}[H]
\centering
    \begin{tabular}{|c|c|}\hline
    \textbf{$\theta$ (degrees)}& \textbf{Moment of inertia $\left(\frac{\mathrm{Kg}}{\mathrm{m^2}}\right)$}\\\hline\hline
    0     & 0.07969 \\  \hline
    22.5  & 0.08206 \\  \hline
    45    & 0.08783 \\  \hline
    67.5  & 0.0936  \\  \hline
    90    & 0.09598 \\  \hline
    112.5 & 0.09359 \\  \hline
    135   & 0.08782 \\  \hline
    157.5 & 0.08206 \\  \hline
    180   & 0.07968 \\  \hline
    202.5 & 0.08206 \\  \hline
    225   & 0.08782 \\  \hline
    247.5 & 0.09359 \\  \hline
    270   & 0.09598 \\  \hline
    292.5 & 0.0936  \\  \hline
    315   & 0.08783 \\  \hline
    337.5 & 0.08206 \\  \hline
    360   & 0.07969 \\  \hline
    \end{tabular}
\caption{Moment of inertia for the pan assembly}
\label{ta:Inertia_PAN}
\end{table}

\begin{equation}\label{eq:I_pan}
    I_{Pan} \approx \left(0.00815\cdot\sin\left(0.0349\cdot \theta + 4.713\right)+0.08783\right) \frac{\mathrm{Kg}}{m^2}
\end{equation}

\section{System restrictions}

\end{document}