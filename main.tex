\documentclass[11pt]{article}
% Language
\usepackage[utf8]{inputenc}
\usepackage[english]{babel}
\babelhyphenation[english]{feed-back-pro-ces-ser}

% Page Formatting
\usepackage[a4paper, width=150mm, top=25mm, bottom=25mm, includefoot]{geometry}

% Line Height
\usepackage{setspace}
\linespread{1.3}

% Paragraph Indentation
\setlength{\parindent}{0pt}

% Header
\usepackage{fancyhdr}
\pagestyle{fancy}
\lhead{}
\chead{}
\makeatletter
\rhead{\slshape{\nouppercase{\rightmark}}}
\makeatother

% Footnote
\usepackage[bottom]{footmisc}
\emergencystretch=1em
\usepackage{scrextend}
\addtokomafont{labelinglabel}{\textnormal}
\usepackage{pdfpages}
\usepackage{needspace}

% Math & Symbols
\usepackage{float}
\usepackage{amsmath}
\usepackage{siunitx}
\usepackage{esvect}
\usepackage{verbatim}

% Equation Formatting
\setlength{\abovedisplayskip}{1em}
\setlength{\belowdisplayskip}{1em}

% Enumeration & Tables
\usepackage{enumerate}
\usepackage{enumitem}
\usepackage{multirow} 
\usepackage{multicol}
\usepackage{ltablex}
\usepackage{spreadtab}
\usepackage{booktabs}
\usepackage{tabto}

% Enumeration Formatting
\newcommand{\tabitem}{~~\llap{\textbullet}~~}
\newcommand{\sqrbulletsml}{\textcolor{black}{\raisebox{.45ex}{\rule{.6ex}{.6ex}}}}
\newcommand{\sqrbulletmed}{\textcolor{black}{\raisebox{.40ex}{\rule{.7ex}{.7ex}}}}

\renewcommand{\labelitemi}{\sqrbulletmed}
\renewcommand{\labelitemii}{\sqrbulletsml}
\renewcommand{\labelitemiii}{\sqrbulletsml}
\renewcommand{\labelitemiv}{\sqrbulletsml}

% Colors
\usepackage{xcolor}

% Color Definitions (latexcolor.com)
\definecolor{cerulean}{rgb}{0.0, 0.48, 0.65}
\definecolor{earthyellow}{rgb}{0.88, 0.66, 0.37}
\definecolor{darkmagenta}{rgb}{0.55, 0.0, 0.55}
\definecolor{darkolivegreen}{rgb}{0.33, 0.42, 0.18}
\definecolor{codegray}{gray}{0.9}
%\newcommand{\code}[1]{\colorbox{codegray}{\texttt{#1}}}

% Code
\usepackage{algorithm}
\usepackage[noend]{algpseudocode}
\usepackage{listings}
\renewcommand{\arraystretch}{1.2}
\renewcommand{\tabcolsep}{0.2cm}

% Code Snippet Formatting
\lstset{
    backgroundcolor=\color{black!5},        % set backgroundcolor
    basicstyle=\footnotesize\ttfamily,      % basic font setting
    frame=single,                           % draw a frame at the top and bottom of the code block
    framesep=5pt,                           % frame margin
    xleftmargin=5pt,                        % frame margin
    xrightmargin=5pt,                       % frame margin
    tabsize=4,                              % tab space width
    showstringspaces=false,                 % don't mark spaces in strings
    breaklines=true,                        % wrap lines
    commentstyle=\color{darkolivegreen},    % comment color
    keywordstyle=\color{darkmagenta},       % keyword color
    stringstyle=\color{earthyellow},        % string color
    identifierstyle=\color{cerulean}
}


% Bibliography & Citation
\usepackage{csquotes}
\usepackage{hyperref}
\usepackage[backend=biber, sorting=none, block=ragged, citestyle=authoryear]{biblatex}
\DeclareNameAlias{sortname}{last-first}
\DeclareNameAlias{default}{last-first}
\addbibresource{resources.bib}

% Continuous Number (1, 2, 3 ..)
\usepackage{chngcntr}
\counterwithout{figure}{chapter}
\counterwithout{table}{chapter}
\counterwithout{equation}{chapter}

% Graphics
\usepackage{graphicx}
\usepackage{pgfplots}
\usepackage{tikz}
\usepackage{circuitikz}
\usetikzlibrary{shapes,arrows}

\usetikzlibrary{shapes,arrows}
\tikzstyle{decision} = [diamond, draw, fill=blue!20, 
    text width=4.5em, text badly centered, node distance=3cm, inner sep=0pt]
\tikzstyle{block} = [rectangle, draw, fill=blue!20, 
    text width=5em, text centered, rounded corners, minimum height=4em]
\tikzstyle{line} = [draw, -latex']
\tikzstyle{cloud} = [draw, ellipse,fill=red!20, node distance=3cm,
    minimum height=2em, text badly centered, text width=4.5em]

\makeatother

\title{RB-PRO4}
\author{Christian}
\date{February 2019}

\begin{document}
\pagenumbering{Roman}

\maketitle

\newpage
\tableofcontents

\newpage
\pagenumbering{arabic}

\section{Introduction}
This is an intruction with a citation\footcite[]{book}, which is pretty nice.

\section{Embedded Software Design}
1. Processes vs Threads
- Limited Memory vs Synkronisering Extras
    - Limiited memory therfor Threads are chosen. -- ...
    - No nested processes (no threads in processes)
- Independent Units vs Event Driven Programs
    - Indepedent units, because a event driven programs are hard to realise as RTOS.
    - Small project
    - Low requirement specifications

2. Thread usages
- Multi-threading and Parallel programming
    - Our CPU supports only 1 "real" thread at a time.
    - We do need / chosen several Threads in this project to realise it.
- Hand-in-hand blocking \& execution
    - We need systemblocking as it is better than to race.
- Code \& Libraries to be re-entrant \& thread-safe
    - It is required thread-safe.
- PThread Library
  - Model cases
    - Producer - Consumer --
    - Peer ( multiple I/O processing) -- SPI take care
    - Need to choose a Library -
    - FREERTOS.org - https://www.freertos.org/RTOS.html

3. Scheduler \& Prio
- Soft Real Time vs Hard Real time
    - Choose which events or processes or tasks are which
    - Nice vs Prio
- Time slicing vs Starvation
    - RTOS tasks are to be run.. but soft real time might consider time slicing due to starvation.
- Sync in Priobased Scheduling
    - ....
- Deadlock Avoidance (bankers)
    - Code must be tested for deadlocks
- Priority Inversion - prio inheritance, prio protect
    - Might not happend, we will design so this will not occur otherwise library for this..

5. Signals vs polling
    - Small systems no need for signals - and indepedent units - therefor polling
- Async vs Sync
    - Async UART, SPI,
- Racing vs Blocking - A race condition is a flaw that occurs when the timing or ordering of events affects
    - Blocking
- Timers vs loops
    - Timers if nes otherwise loops.

6. Interrupts vs Signals
    - No interrupts in generel bcs no reason to use it!

7. Sockets
- Interface to multiple Devices
    - 1 UART to PC
    - 3 SPI per purpose FPGA.
    - Joystick
        - But no common bus

8. Fejl Tolerancer
- Default ErrorHandler / ExceptionHandler
    - blinking LED + UART code
- Avoid Race Conditions w/ proper sync
- Avoid Memory Leaks w/ Memory usage balancing - Verify memory profilers
- Handle error cases, Extreme conditions
    - Defaults for a switch, else, if
- Avoid infinite  loops - use timeouts.
- Regorganize code to avoid / prevent deadlocks

9. Design for Crash Recovery
- Restart crusial flows - Main task f.eks.
- Design for Hang Recovery - with Watchdogs
- Design for power fialtures - brown out detections / power-down saves

Design for Hardware
- Speed Mismatch (faster/slower)
- Time-sensitive Hardware interaction spi, i2s - precise busy wait delays
Non-responsive hardware
- Timeout rather than uncondictional accesses
 Hardware with irregular/incorrect responses
- Do prober handling for the extreme / abnormal / invalid cases
Hardware delays (Debounce, Rise time)
- Appropiate timers / delays to offset

Design for Performance
- IPC vs Sync
    - Multiprocesses (Context Switches) vs Multihreaded ( same context)
    - Communication vs Sync overheads
- Schedulers and Priority
    - Normal vs Real time
    - Nice vs Priority
- Yielding vs Busy wait
    - LAtency vs Precision
- Interrupt / signal usage
    - High Prio and Async attention
    - Missing = invalidate assumptions
- Design for Real time Performance
- Have very few real time processes
- Keep the real time short \& effiecient
    - INGEN MEMORY ALLOCATION
    - if essential move to init sections
-
\newpage
%\subfile{pages/lipsum.tex}

\newpage
\printbibliography
\end{document}
