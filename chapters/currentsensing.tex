\subsection{Current sensing}


there are most often used either ohms law or magnet field current sensing. The simple way is using ohms law, and is therefore chosen. This way uses a low shunt resistor and the voltage across it is converted to a current, therefore requiring an op-amp to multiply the voltage. There are 3 places the shunt resistor can be placed, all of them got pros and cons. It is usally prefered to sense as close to the power supply as possible, therefore high-side measuring is usally used. Since the motor has friction, filtering has to be used to reduce electromagnetic interference and spikes. A simple technique to do this is using a common mode filter.... The common mode filter act as a transmission line for differential mode signals and as inductors for common mode noise. Besides using a common mode filter two low pass filters are used..... 
The circuit parameters:

\[
R_{sense}\ll R_{1}\;\&\&\;R_{2}
\]
Besides that it is important $R_{2}=R_{1}$ is close to eachother,
otherwise the CMRR is will degrade.

\[
C_{1}=C_{2}
\]
It is not as important that the capacitors are equal, but around 5\%
accuracy.

\[
C_{3}\gg C_{1}\;and\;C_{3}\gg C_{1}
\]
The common mode filter can be calculated:

\[
f_{-3db}=\frac{1}{2\pi R_{1}C_{1}}\;and\;f_{-3db}=\frac{1}{2\pi R_{2}C_{2}}
\]
The differential mode filter

\[
f_{-3db}=\frac{1}{2\pi\left(R_{1}+R_{2}\right)\left[\left(\frac{C_{1}\cdot C_{2}}{C_{1}+C_{2}}\right)+C_{3}\right]}
\]

\begin{tabular}{|c|}
\hline 
The common mode refers to signals or noise that flow in the same direction
in a pair of lines.\tabularnewline
\hline 
\hline 
The differential (normal) mode refers to signals or noise that flow
in opposite directions in a pair of lines.\tabularnewline
\hline 
\end{tabular}