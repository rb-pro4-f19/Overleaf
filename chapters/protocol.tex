\subsection{SPI Protocol}
Communication between the MCU (master) and FPGA (slave) is done by the use of SPI. A custom 16-bit frame format is used. The MCU can transmit a frame of type \texttt{send/request} while the FPGA can reply with a frame of type \texttt{response}.\medskip

\subsubsection{MCU}
The standard Freescale protocol is configured to \texttt{16-bit} frames with transmission rate of \texttt{8 Mb/s} in \texttt{Mode 0}. The send/request type frames are comprised of an address, data and checksum field of the sizes: respectively \texttt{ADDR:4}, \texttt{DATA:8} and \texttt{CHKSUM:4}. Response frames omit the address field and consist instead of respectively \texttt{DATA:12} and \texttt{CHKSUM:4}.\medskip

Send/request type frames must be acknowledged with a frame of valid checksum, although the content of the frame may be disregarded. Checksum is calculated using the BSD algorithm\footcite[]{bsd_wiki} on the 12 most-significant bits of a frame.\medskip

\begin{table}[ht!]
\centering
\begin{tabular}{|c|c|c|c|}
\hline
\multicolumn{1}{|l|}{} & Address & Data     & Checksum \\ \hline
CTRL                   & 0000    & XXXXXXXX & XXXX     \\ \hline
PWM1                   & 0001    & XXXXXXXX & XXXX     \\ \hline
PWM2                   & 0010    & XXXXXXXX & XXXX     \\ \hline
ENC1                   & 0011    & XXXXXXXX & XXXX     \\ \hline
ENC2                   & 0100    & XXXXXXXX & XXXX     \\ \hline
HAL1                   & 0101    & XXXXXXXX & XXXX     \\ \hline
HAL2                   & 0110    & XXXXXXXX & XXXX     \\ \hline
AMP1                   & 0111    & XXXXXXXX & XXXX     \\ \hline
AMP2                   & 1000    & XXXXXXXX & XXXX     \\ \hline
\end{tabular}
\caption{Combinations of a send/request frame.}
\label{tab: spi_1}
\end{table}

Response is:

\begin{table}[h!]
\centering
\begin{tabular}{cccc}
\hline
\multicolumn{1}{|l|}{}       & \multicolumn{1}{c|}{Data}            & \multicolumn{1}{c|}{Checksum} \\ \hline
\multicolumn{1}{|c|}{CTRL}   & \multicolumn{1}{c|}{0000000000000}   & \multicolumn{1}{c|}{0000}     \\ \hline
\multicolumn{1}{|c|}{PWM1}   & \multicolumn{1}{c|}{0000000000000}   & \multicolumn{1}{c|}{0000}     \\ \hline
\multicolumn{1}{|c|}{PWM2}   & \multicolumn{1}{c|}{0000000000000}   & \multicolumn{1}{c|}{0000}     \\ \hline
\multicolumn{1}{|c|}{ENC1}   & \multicolumn{1}{c|}{XXXXXXXXXXXX}    & \multicolumn{1}{c|}{XXXX}     \\ \hline
\multicolumn{1}{|c|}{ENC2}   & \multicolumn{1}{c|}{XXXXXXXXXXXX}    & \multicolumn{1}{c|}{XXXX}     \\ \hline
\multicolumn{1}{|c|}{HALL1}  & \multicolumn{1}{c|}{XXXXXXXXXXXX}    & \multicolumn{1}{c|}{XXXX}     \\ \hline
\multicolumn{1}{|c|}{HALL2}  & \multicolumn{1}{c|}{XXXXXXXXXXXX}    & \multicolumn{1}{c|}{XXXX}     \\ \hline
\multicolumn{1}{|c|}{CUR1}   & \multicolumn{1}{c|}{XXXXXXXXXXXX}    & \multicolumn{1}{c|}{XXXX}     \\ \hline
\multicolumn{1}{|c|}{CUR2}   & \multicolumn{1}{c|}{XXXXXXXXXXXX}    & \multicolumn{1}{c|}{XXXX}     \\ \hline
\end{tabular}
\caption{Response frame.}
\end{table}

\subsubsection{-...-}
is designed with 3 SPI modules: a reciever, transmitter and topmodule. The topmodule is made to ease the task of the controller by a simple interface; checksum validation and easy flow management (will discuss this later). The transmitter and recieving module is following the MCU's protocol, which is the Freescale protocol. The overall flow in the FPGA is respectively: new data is coming, validate data, send data to controller, get reply from controller and reply on next message.

\subsubsection{Transmitter/Reciever SPI}
those module are designed with 1 main state machine containing the following states: \texttt{IDLE}, \texttt{TRANSMIT}, \texttt{WAITING}. The major design choice was to pick asynchronous or synchronous design (use internal clk to sample incoming data). Pros for asynchronous is that it has minimum delaytime and this is baiscally it, which is why synchronous is used. The synchronous method uses the internal clk and shift registers for incoming data and therefore debounces the signal. It is only slightly slower as long as the debouncing is kept low. The module flow is when slaveselect is low, follow the freescale protocol. When slaveselect goes high reciever delivers new data to topmodule SPI and transmitter is ready to send new data on next low slaveselect.

\subsubsection{Topmodule SPI}
this module is designed to recieve new data, validate it with a 12bit BCD checksum, send the information to the controller, the controller parses the information to the module it is supposed to communicate with. Then the module acknowlegedes to the controller and the message is then parsed to the spi module, where a checksum is added as a trailer and ready to sent on next clock. The main concern of designing this module is based on the clock cycles from 







