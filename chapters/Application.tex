Dette projekt omhandler opsætning af et Pan/tilt system. Systemet har 2 frihedsgrader, hhv pan og tilt. Systemet besår ligeledes af 2 programmerbare enheder; en FPGA og en MIC. FPGA’en formål er at skabe en direkte forbindelse til diverse sensorer som, encoder, hallsensorer og eventuelle currentsensorer. Ligeledes er det FPGA’ens opgave at generere et PWM signal for til at styre to motorerne, og derigennem systemets 2 frihedsgrader. De datapunkter FPGA’en indsamler kommunikeres videre til MIC’en gennem en SPI-protokol. MIC’en formål er at beregne motorernes hastighed, på baggrund af data indsamlet gennem FPGA’en. Ud fra disse data, skal MIC’en være i stand til at benytte en PID-controller for at kunne regulere motorer positionerne til den ønskede værdi. Den ønskede værdi vil i dette projekt fastsættes ud fra et joysticks position. Det benyttet joystick outputter 2 værdier der således kan mappes til motorens positioner. Derved skaber MIC’en sammen med FPGA’en en overgang mellem et joystick og Pan/tilt systemet.